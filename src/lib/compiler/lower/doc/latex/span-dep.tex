\section{Span Dependency Resolution} \label{sec:span-dep}

The span dependency resolution phase is used to resolve the values of
client defined \href{constants.html}{constants} and \href{labels.html}{labels}
in a program.  An instruction whose immediate operand field contains a
constant or \href{labexp.html}{label expression} which
is too large is rewritten into a sequence of instructions to compute
the same result.  Similarly, short branches referencing labels that are 
too far are rewritten into the long form.   For architectures
that require the filling of delay slots, this is performed at the same
time as span depedency resolution, to ensure maximum benefit results.

\subsubsection{The Interface}

The api \sml{Span_Dependent_Jumps} describes
architectural information about span dependence resolution.

\begin{SML}
api \mlrischref{span/span-dependent-jumps.api}{Span_Dependent_Jumps} = sig
  package i : \href{instructions.html}{Instruction_Set}
  package c : \href{cells.html}{Cells}
    sharing I.C = C

  my branchDelayedArch : Bool
  my isSdi : I.instruction -> Bool
  my minSize : I.instruction -> int
  my maxSize : I.instruction -> int
  my sdiSize : I.instruction * (C.cell -> C.cell)
                              * (Label.label -> int) * int -> int
  my expand : I.instruction * int * int -> I.instruction list
end
\end{SML}

The components in this interface are:
\begin{description}
  \item[branchDelayedArch] A flag indicating whether the architecture
contains delay slots.  For example, this would be true on the 
Sparc but would be false on the x86.
   \item[isSdi] This function returns true if the instruction is 
\newdef{span dependent}, i.e.~its size depends either on some unresolved
constants, or on its position in the code stream.
   \item[sdiSize]  This function takes a span dependent instruction, 
a \href{regmap.html}{regmap},
a mapping from labels to code stream position, and 
its current code stream position and returns the size of its
expansion in bytes.
   \item[expand] This function takes a span dependent instruction,
its size, and its location and return its expansion.
\end{description}

The api \sml{Basic_Block_Scheduler} is the api of the phase that performs
span depedennce resolution and code generation.
\begin{SML}
api \mlrischref{span/basic-block-scheduler.api}{Basic_Block_Scheduler} = sig
  package f : \href{cluster.html}{FLOWGRAPH}

  my bbsched : F.cluster -> Void
  my finish : Void -> Void
  my clean_up : Void -> Void
end
\end{SML}

\subsubsection{The Modules}

Three different generics are present in the \MLRISC{} system for
performing span dependence resolution and code generator.
Generic \sml{basic_block_scheduler2_g} is the simplest one, which does not perform
delay slot filling.
\begin{SML}
generic package basic_block_scheduler2_g
  (package flowgraph : \mlrischref{cluster/flowgraph.sig}{FLOWGRAPH}
   package jumps : \mlrischref{span/span-dependent-jumps.api}{Span_Dependent_Jumps}
   package Emitter : \href{mc.html}{Instruction_Emitter}
     sharing Emitter.P = flowgraph.P
     sharing flowgraph.I = Jumps.I = Emitter.I
  ): Basic_Block_Scheduler 
\end{SML}

Generic \sml{span_dependency_resolution_g} performs both span dependence
resolution and delay slot filling at the same time.
\begin{SML}
generic package span_dependency_resolution_g
  (package flowgraph : \mlrischref{cluster/flowgraph.sig}{FLOWGRAPH}
   package Emitter : \href{mc.html}{Instruction_Emitter}
   package jumps : \mlrischref{span/span-dependent-jumps.api}{Span_Dependent_Jumps}
   package DelaySlot : \href{delayslots.html}{Delay_Slot_Properties}
   package props : \mlrischref{instruction/instructionProps.sig}{Instruction_Properties}
     sharing flowgraph.P = Emitter.P
     sharing flowgraph.I = Jumps.I = DelaySlot.I = Props.I = Emitter.I
  ) : Basic_Block_Scheduler 
\end{SML}

Finally, generic package \sml{x86_span_dependency_resolution_g} is a span dependency resolution
module specially written for the \href{x86.html}{x86} architecture.
\begin{SML}
generic package x86_span_dependency_resolution_g
  (package code_string : \mlrischref{emit/code-string.api}{Code_String}
   package jumps: \mlrischref{span/span-dependent-jumps.api}{Span_Dependent_Jumps}
   package props : \mlrischref{instruction/instructionProps.sig}{Instruction_Properties}
   package Emitter : \mlrischref{span/vlBatchPatch.sig}{Machine_Code_Emitter}
   package flowgraph : \href{cluster.html}{FLOWGRAPH}
   package asm : \href{asm.html}{Instruction_Emitter}
      sharing Emitter.I = Jumps.I = flowgraph.I = Props.I = Asm.I) : Basic_Block_Scheduler 
\end{SML}


