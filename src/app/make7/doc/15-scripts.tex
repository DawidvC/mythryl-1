% -*- latex -*-

\section{Auxiliary scripts}

\subsection{Building stand-alone programs}
\label{sec:build-an-executabe-heap-image}

The programmer should normally have no need to invoke {\tt
make7.build_executable_heap_image} (see Section~\REF{sec:build-an-executable-mythryl-heap-image:support}) directly.
Instead, we provide the script {\tt bin/build-an-executable-mythryl-heap-image} which does all the
work.  To be able to use {\tt build-an-executable-mythryl-heap-image}, one must implement a library
exporting a package that has some function suitable to be an
argument to {\tt lib7.spawn_to_disk}.  Suppose the library is called
{\tt myproglib.make6}, the package is called {\tt MyProg}, and the
function is called {\tt MyProg.main}.  If one wishes to produce a heap
image file {\tt myprog} one simply has to invoke the following
command:

\begin{verbatim}
  bin/build-an-executable-mythryl-heap-image myproglib.make6 MyProg.main myprog
\end{verbatim}

The heap image is written only if needed: if a heap image exists and
is newer than all source code involved, provided that none of the
sources have to be recompiled, {\tt build-an-executable-mythryl-heap-image}
will just issue a message indicating that everything is up-to-date.

As in the case of {\tt mythryld}, it is possible to define or undefine
preprocessor symbols using {\tt -D} or {\tt -U} options (see
Section~\REF{sec:cmdline:defundef}).  These options must be specified
before the three regular arguments.  Thus, the full command line
syntax is:

\begin{verbatim}
  bin/build-an-executable-mythryl-heap-image [DU-options] myproglib.make6 MyProg.main myprog
\end{verbatim}

\subsubsection{Bootstrapping: How {\tt build-an-executable-mythryl-heap-image} works}

Internally, {\tt build-an-executable-mythryl-heap-image} generates a temporary wrapper library
containing a single call of {\tt lib7.spawn_to_disk} as part of the
library's module-initialization code.  Once this is done, mythryld is
started, {\tt make7::build_executable_heap_image} is invoked (with the main project
description file, the generated wrapper library file, and the heap
image name as arguments), and a {\tt OH7_FILES_TO_LOAD} file is written.
If all these steps were successful, {\tt build-an-executable-mythryl-heap-image}
invokes {\tt bin/runtime7} with the {\tt --runtime-o7-files-to-load} and
{\tt --runtime-heap} commandline options specified, causing it to load and link the
desired {\tt .o7} files per {\tt OH7_FILES_TO_LOAD} and then
write the resulting ``executable'' (heap image) with {\tt #!/usr/bin/runtime7} shebang line)
to the given output file, here {\em myprog}.

Each line of the {\tt OH7_FILES_TO_LOAD} file specifies one module to be linked
into the final heap image.  The runtime system reads these
lines one-by-one, loads the corresponding modules, and executes their
initialization code.  Since the last module has been arranged (by way
of using the wrapper library from above) to contain a call of {\tt
lib7.spawn_to_disk}, initialization of this module causes the program's
heap image to be written and the bootstrap procedure to terminate.

\subsection{Generating dependencies for {\tt make}}
\label{sec:makedepend}

When Mythryl7 programs are used as parts of larger projects, it can become
necessary to use make7 (or, e.g., {\tt build-an-executable-mythryl-heap-image}
as described in Section~\REF{sec:build-an-executable-mythryl-heap-image})
in a traditional makefile for Unix' {\bf make}.
To avoid repeated invocations, the dependency information that
make7 normally manages internally must be described externally so that
{\bf make} can process it.

For this purpose, it is possible to let make7's dependency analyzer
generate a list of files that a given program depends on (see
Section~\REF{sec:makedepend:support}).  The {\tt makedepend7}
scripts conveniently wraps this functionality in such a way that it
resembles the familiar {\bf makedepend} facility found on many Unix
installations for the use by C projects.

An invocation of {\tt makedepend7} takes two mandatory arguments:
the root {\tt .make6} file of the program in question and the name
of the target that is to be used by the generated makefile entry.
Thus, a typical command line has the form:

\begin{verbatim}
  makedepend7 project.make6 targetname
\end{verbatim}

This will cause {\tt makedepend7} to first look for a file named
{\tt makefile} and if that cannot be found for {\tt Makefile}.  (An
error message is issued if neither of the two exists.)  After deleting
any previously generated entry for this description-target
combination, the script will invoke make7 and add up-to-date dependency
information to the file.

Using the {\tt -f} option it is possible to force an arbitrary
programmer-specified file to be used in place of {\tt makefile} or
{\tt Makefile}.

Some of the files a make7-managed program depends on are stable
libraries.  Since the file names for stable libraries vary according
to current CPU architecture and operating system, writing them
directly would require different entries for different systems.  To
avoid this problem (most of the time\footnote{The careful reader may
have noticed that because of make7's conditional compilation it is
possible that dependency information itself varies between different
architectures or operating systems.}), {\tt makedepend7} will use
{\bf make}-variables {\tt \$(ARCH)} and {\tt \$(OPSYS)} as
placeholders within the information it generates.  It is the
programmer's responsibility to make sure that these variables are set
to meaningful values at the time {\bf make} is eventually being
invoked.  This feature can be turned off (causing actual file names to
be used) by specifying the {\tt -n} option to {\tt makedepend7}.

In cases where the programmer prefers other strings to be used in
place of {\tt \$(ARCH)} or {\tt \$(OPSYS)} (or both) one can specify
those strings using the {\tt --a} and {\tt -o} options, respectively.

Like {\tt make7} (Section~\REF{sec:make7})
(Section~\REF{sec:cmdline:defundef}), the {\tt makedepend7} command
also accepts {\tt -D} and {\tt -U} command line options.

Thus, the full command line syntax for {\tt makedepend7} is:

\begin{verbatim}
  makedepend7 [DU-options] [-n] [-f makefile] project.make6 target
  makedepend7 [DU-options] [-a arch] [-o os] [-f makefile] project.make6 target
\end{verbatim}

(If {\tt -n} is given, then any additional {\tt -a} or {\tt -o}
options---while not illegal---will be ignored.)
