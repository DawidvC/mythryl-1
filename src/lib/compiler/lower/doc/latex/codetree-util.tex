\section{Codetree Utilities} 

The \LOWCODE{} system contains numerous utilities for working with
Codetree datatypes.  Some of the following utilizes are also useful for clients
use:
\begin{description}
  \item[codetree_utils] implements basic hashing, equality and pretty
printing functions,
  \item[codetree_fold] implements a fold function over the Codetree datatypes,  
  \item[codetree_rewrite] implements a generic rewriting engine,
  \item[codetree_simplify] implements a simplifier that performs algebraic
simplification and constant folding.
\end{description}
\subsubsection{Hashing, Equality, Pretty Printing}

The generic package \lowcodehref{codetree/codetree-utils.pkg}{codetree_utils} provides
the basic utilities for hashing an Codetree term, comparing two
Codetree terms for equality and pretty printing.  The hashing and comparision
functions are useful for building hash tables using Codetree enum as keys.
The api of the generic is:
\begin{SML}
api \lowcodehref{codetree/codetree-utils.api}{Codetree_Utilities} =
sig
   package t : Codetree 

   /*
    * Hashing
    */
   my hashStm   : T.statement -> word
   my hashRexp  : T.int_expression -> word
   my hashFexp  : T.float_expression -> word
   my hashCCexp : T.bool_expression -> word

   /*
    * Equality
    */
   my eqStm     : T.statement * T.statement -> Bool
   my eqRexp    : T.int_expression * T.int_expression -> Bool
   my eqFexp    : T.float_expression * T.float_expression -> Bool
   my eqCCexp   : T.bool_expression * T.bool_expression -> Bool
   my eqLowcodes : T.lowcode list * T.lowcode list -> Bool

   /*
    * Pretty printing 
    */
   my show : (String list * String list) -> T.printer

   my stmToString   : T.statement -> String
   my rexpToString  : T.int_expression -> String
   my fexpToString  : T.float_expression -> String
   my ccexpToString : T.bool_expression -> String

end
generic package \lowcodehref{codetree/codetree-utils.pkg}{codetree_utils} 
  (package t : Codetree
   #  Hashing extensions 
   my hashSext  : T.hasher -> T.sext -> word
   my hashRext  : T.hasher -> T.rext -> word
   my hashFext  : T.hasher -> T.fext -> word
   my hashCCext : T.hasher -> T.ccext -> word

   #  Equality extensions 
   my eqSext  : T.equality -> T.sext * T.sext -> Bool
   my eqRext  : T.equality -> T.rext * T.rext -> Bool
   my eqFext  : T.equality -> T.fext * T.fext -> Bool
   my eqCCext : T.equality -> T.ccext * T.ccext -> Bool

   #  Pretty printing extensions 
   my showSext  : T.printer -> T.sext -> String
   my showRext  : T.printer -> T.ty * T.rext -> String
   my showFext  : T.printer -> T.fty * T.fext -> String
   my showCCext : T.printer -> T.ty * T.ccext -> String
  ) : Codetree_Utilities =
\end{SML} 

The types \sml{hasher}, \sml{equality},
and \sml{printer} represent functions for hashing,
equality and pretty printing.   These are defined as:
\begin{SML} 
   type hasher =
      \{statement    : T.statement -> word,
       int_expression   : T.int_expression -> word,
       float_expression   : T.float_expression -> word,
       bool_expression  : T.bool_expression -> word
      \}    

   type equality =
      \{ statement    : T.statement * T.statement -> Bool,
        int_expression   : T.int_expression * T.int_expression -> Bool,
        float_expression   : T.float_expression * T.float_expression -> Bool,
        bool_expression  : T.bool_expression * T.bool_expression -> Bool
      \} 
   type printer =
      \{ statement    : T.statement -> String,
        int_expression   : T.int_expression -> String,
        float_expression   : T.float_expression -> String,
        bool_expression  : T.bool_expression -> String,
        dstReg : T.ty * T.var -> String,
        srcReg : T.ty * T.var -> String
      \}
\end{SML}

For example, to instantiate a \sml{Utils} module for our \sml{DSPCodetree},
we can write:
\begin{SML}
   package u = codetree_utils
     (package t = DSPCodetree
      fun hashSext \{statement, int_expression, float_expression, bool_expression\} (FOR(i, a, b, s)) =
           unt.fromIntX i + int_expression a + int_expression b + statement s
      and hashRext \{statement, int_expression, float_expression, bool_expression\} e =
          (case e of
             SUM(i,a,b,c) => unt.fromIntX i + int_expression a + int_expression b + int_expression c
           | SADD(a,b) => int_expression a + int_expression b
           | SSUB(a,b) => 0w12 + int_expression a + int_expression b
           | SMUL(a,b) => 0w123 + int_expression a + int_expression b
           | SDIV(a,b) => 0w1245 + int_expression a + int_expression b
          )
      fun hashFext _ _ = 0w0
      fun hashCCext _ _ = 0w0
      fun eqSext \{statement, int_expression, float_expression, bool_expression\} 
        (FOR(i, a, b, s), FOR(i', a', b', s')) =
           i=i' and int_expression(a,a') and int_expression(b,b') and statement(s,s')
      fun eqRext \{statement, int_expression, float_expression, bool_expression\} (e,e') =
       (case (e,e') of
          (SUM(i,a,b,c),SUM(i',a',b',c')) => 
            i=i' and int_expression(a,a') and int_expression(b,b') and statement(c,c')
        | (SADD(a,b),SADD(a',b')) => int_expression(a,a') and int_expression(b,b')
        | (SSUB(a,b),SSUB(a',b')) => int_expression(a,a') and int_expression(b,b')
        | (SMUL(a,b),SMUL(a',b')) => int_expression(a,a') and int_expression(b,b')
        | (SDIV(a,b),SDIV(a',b')) => int_expression(a,a') and int_expression(b,b')
        | _ => false
       )
      fun eqFext _ _ = true
      fun eqCCext _ _ = true

      fun showSext \{statement, int_expression, float_expression, bool_expression, dstReg, srcReg\}  
            (FOR(i, a, b, s)) =
          "for("^dstReg i^":="^int_expression a^".."^int_expression b^")"^statement s
      fun ty t = "."^int.to_string t
      fun showRext \{statement, int_expression, float_expression, bool_expression, dstReg, srcReg\} e = 
           (case (t,e) of
             SUM(i,a,b,c) => 
              "sum"^ty t^"("^dstReg i^":="^int_expression a^".."^int_expression b^")"^int_expression c
           | SADD(a,b) => "sadd"^ty t^"("int_expression a^","^int_expression b^")"
           | SSUB(a,b) => "ssub"^ty t^"("int_expression a^","^int_expression b^")"
           | SMUL(a,b) => "smul"^ty t^"("int_expression a^","^int_expression b^")"
           | SDIV(a,b) => "sdiv"^ty t^"("int_expression a^","^int_expression b^")"
           )
      fun showFext _ _ = ""
      fun showCCext _ _ = ""
     )
\end{SML}

\subsubsection{Codetree Fold}
The generic package \lowcodehref{codetree/codetree-fold.pkg}{codetree_fold}
provides the basic functionality for implementing various forms of
aggregation function over the Codetree datatypes.  Its api is
\begin{SML}
api \lowcodehref{codetree/codetree-fold.api}{Codetree_Fold} =
sig
   package t : Codetree

   my fold : 'b folder -> 'b folder
end
generic package \lowcodehref{codetree/codetree-fold.pkg}{codetree_fold}
  (package t : Codetree
   #  Extension mechnism 
   my sext  : 'b T.folder -> T.sext * 'b -> 'b
   my rext  : 'b T.folder -> T.ty * T.rext * 'b -> 'b
   my fext  : 'b T.folder -> T.fty * T.fext * 'b -> 'b
   my ccext : 'b T.folder -> T.ty * T.ccext * 'b -> 'b
  ) : Codetree_Fold =
\end{SML}
The type \newtype{folder} is defined as:
\begin{SML}
   type 'b folder =
       \{ statement   : T.statement * 'b -> 'b,
         int_expression  : T.int_expression * 'b -> 'b,
         float_expression  : T.float_expression * 'b -> 'b, 
         bool_expression : T.bool_expression * 'b -> 'b
       \}
\end{SML}


\subsubsection{Codetree Rewriting}

The generic package \lowcodehref{codetree/codetree-rewrite.pkg}{codetree_rewrite}
implements a generic term rewriting engine which is useful for performing
various transformations on Codetree terms. Its api is
\begin{SML}
api \lowcodehref{codetree/codetree-rewrite.api}{Codetree_Rwrite} =
sig
   package t : Codetree

  my rewrite : 
       #  User supplied transformations 
       \{ int_expression  : (T.int_expression -> T.int_expression) -> (T.int_expression -> T.int_expression), 
         float_expression  : (T.float_expression -> T.float_expression) -> (T.float_expression -> T.float_expression),
         bool_expression : (T.bool_expression -> T.bool_expression) -> (T.bool_expression -> T.bool_expression),
         statement   : (T.statement -> T.statement) -> (T.statement -> T.statement)
       \} -> T.rewriters
end
generic package \lowcodehref{mltre/codetree-rewrite.pkg}{codetree_rewrite}
  (package t : Codetree
   #  Extension 
   my sext : T.rewriter -> T.sext -> T.sext
   my rext : T.rewriter -> T.rext -> T.rext
   my fext : T.rewriter -> T.fext -> T.fext
   my ccext : T.rewriter -> T.ccext -> T.ccext
  ) : Codetree_Rwrite =
\end{SML}

The type \newtype{rewriter} is defined in api
\lowcodehref{codetree/codetree.api}{Codetree} as:
\begin{SML}
   type rewriter = 
       \{ statement   : T.statement -> T.statement,
         int_expression  : T.int_expression -> T.int_expression,
         float_expression  : T.float_expression -> T.float_expression,
         bool_expression : T.bool_expression -> T.bool_expression
       \} 
\end{SML}
 
\subsubsection{Codetree Simplifier}

The generic package \lowcodehref{codetree/codetree-simplify.pkg}{CodetreeSimplify}
implements algebraic simplification and constant folding for Codetree.
Its api is:
\begin{SML}
api \lowcodehref{codetree/codetree-simplify.api}{CODETREE_SIMPLIFIER} =
sig

   package t : Codetree

   my simplify  :
       { addressWidth : int } -> T.simplifier
   
end
generic package \lowcodehref{codetree/codetree-simplify.pkg}{codetree_simplifier_g}
  (package t : Codetree
   #  Extension 
   my sext : T.rewriter -> T.sext -> T.sext
   my rext : T.rewriter -> T.rext -> T.rext
   my fext : T.rewriter -> T.fext -> T.fext
   my ccext : T.rewriter -> T.ccext -> T.ccext
  ) : CODETREE_SIMPLIFIER =
\end{SML}

Where type \newdef{simplifier} is defined in api 
\lowcodehref{codetree/codetree.api}{Codetree} as:
\begin{SML}
   type simplifier =
       \{ statement   : T.statement -> T.statement,
         int_expression  : T.int_expression -> T.int_expression,
         float_expression  : T.float_expression -> T.float_expression,
         bool_expression : T.bool_expression -> T.bool_expression
       \}
\end{SML}


